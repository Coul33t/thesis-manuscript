\part{Introduction}
\chapter{Cadre et contexte}  %Title of the First Chapter
Ce chapitre introduit le cadre de recherche de cette thèse, à la croisée des domaines de l'apprentissage humain du mouvement et des Environnements Informatiques pour l'Apprentissage Humain (EIAH). À l'issue d'une mise en contexte, les problématiques de recherche sont soulevées, et la contribution apportée est présentée.\\

L'apprentissage est un élément fondamental de l'humain. Notre capacité à observer, analyser, retenir et reproduire diverses choses nous permet de progresser tout au long de notre vie et de nous perfectionner. Cet apprentissage peut prendre plusieurs formes : formel / informel, par transmission orale, par imitation, mettant en jeu des connaissances tacites/explicites, \textit{etc.} \parencite{Eraut200Nfl}. Les récentes avancées technologiques font que l'informatique est maintenant profondément ancrée dans notre société. L'utilisation conjointe des possibilités offertes par l'informatique et du savoir et de l'expertise humaine dans les processus d'apprentissage permet la création d'environnements informatiques pour soutenir l'enseignement et l'apprentissage humain. En particulier, ces environnements peuvent s'appliquer à la formation aux gestes et aux mouvements.\\

Le mouvement humain peut être vu comme une succession de postures d'un corps dans l'espace, variant au cours du temps. Son apprentissage peut se faire de différentes manières : par imitation, c'est-à-dire la reproduction de chaque posture d'une personne experte démontrant le mouvement (p. ex. apprentissage de la danse), par acquisition des propriétés biomécaniques clés connues au préalable du geste cible (p. ex. coude en bas pour un lancer de pétanque, flexion puis extension des jambes vers l'avant et le haut permettant d'initier le service au tennis) ou par recherche des gestes permettant d'atteindre un état ciblé des objets manipulés par l'intermédiaire des mouvements à apprendre (opération chirurgicale nécessitant la manipulation d'un outil tel qu'un scalpel). L'apprentissage d'un geste peut poser plusieurs contraintes, en fonction du mouvement à effectuer : contraintes matérielles, d'espace, de mobilité, d'environnement, etc. l'objectif de cet apprentissage peut être multiple : caractère moteur (le mouvement) mais aussi un caractère fonctionnel (liée à l'objectif de l'action) ou caractère structurant (liées aux connaissances du sujet et du contexte d'apprentissage).

La capture des mouvements peut se faire à l'aide de nombreux dispositifs, tels que : les caméras RGB-D (\textit{Red Green Blue Depth}), popularisées grâce à l'apparition de systèmes grand publics abordables comme la Kinect de Microsoft, utilisable dans des contextes de recherche où les interactions gestuelles sont " simples " (c'est-à-dire quelques mouvements brefs sur un ensemble restreint d'articulations). Des combinaisons constituées de capteurs inertiels, telles que le Perception Neuron de NOITOM, proposent une capture intégrale du corps humain, tout en ayant un encombrement réduit. Pour des captures de haute précision, l'industrie du cinéma a souvent recours à des dispositifs de captation utilisant des caméras infrarouges, disposées autour de la scène, à l'aide d'une combinaison de marqueurs réfléchissants, portée par les acteurs. En fonction des besoins et du coût, et grâce au large choix des moyens de captation, il est possible d'obtenir des données de mouvements exploitables informatiquement et de construire des EIAH dédiés à leurs apprentissages.

La conception d'un environnement pour apprendre des mouvements nécessite de relever des défis propres aux domaines de l'étude des mouvements et de l'ingénierie des EIAH. Les prochaines sections sont dédiées à la définition des différents éléments constituant le domaine de recherche concerné par ce manuscrit, ainsi qu'à la problématique de la thèse et l'approche proposée.\\

\section{Environnements Informatiques pour l'Apprentissage Humain (EIAH)}
Les Environnements Informatiques pour l'Apprentissage Humain (EIAH) sont des environnements ayant pour but de favoriser l'apprentissage, en aidant, guidant et évaluant les apprenants d'une part et en assistant les enseignants d'autre part : que ce soit en présentiel, à distance, ou en situation mixte \parencite{Tchounikine2009PdR}. Une situation d'apprentissage impliquant un EIAH est en conséquence constituée d'au moins un artefact informatique, d'un enseignant (bien que la connaissance experte puisse être injectée dans l'EIAH), d'un apprenant, du matériel à manipuler (dans le contexte d'un apprentissage de gestes sportifs ou de gestes chirurgicaux par exemple), de l'environnement spécifique où a lieu l'apprentissage (p. ex. étude géologique sur le terrain), etc. Cependant, le point fondamental d'une situation d'apprentissage impliquant l'utilisation d'un EIAH est son aspect pédagogique. Ainsi, un EIAH doit être développé afin de servir et s'adapter à une ou plusieurs situations d'apprentissage. Autour des EIAH s'articulent plusieurs domaines de recherche : informatique, sciences de l'éducation, science de l'information et de la communication, didactique, psychologie, etc. Ces domaines sont complémentaires pour aboutir à la création d'un EIAH supportant de manière efficace le processus d'apprentissage. De plus, la fonction visée de l'EIAH (apprentissage théorique,  pratique, par imitation, sur une ou plusieurs séances, etc.) influe sur les compétences à mettre en œuvre pour son développement. La recherche en informatique sur les EIAH s'intéresse d'une part à la conception d'un tel système, mais également à son implémentation ainsi qu'à l'évaluation des impacts de son utilisation. Nous nous intéressons ici à l'accompagnement de l'expert, lors de sa tâche de transmission de savoir, et de l'apprenant dans sa tâche d'apprentissage. Les travaux présentés dans ce manuscrit n'ont pas vocation à remplacer l'expert lors de l'apprentissage, mais à l'assister dans la tâche d'observation, d'analyse et de retours donnés à l'apprenant.

\section{Le mouvement humain}
\subsection{Définition}
Le mouvement entendu comme le déplacement d'un corps dans l'espace à un moment donné, par rapport à un référentiel, est composé de variations à la fois spatiales et temporelles. Ce sont les différents paramètres de ces variations qui vont permettre de différencier un mouvement d'un autre, et leur donner une sémantique. En effet, le mouvement humain est porteur d'informations : il est possible d'extraire des informations bas-niveau, liées à la cinématique et la dynamique \parencite{Nunes2016}. Le geste est également porteur de sens dans le contexte de la communication verbale \parencite{Huang2015}, ou non-verbale \parencite{Chang201379}. En plus de cela, il est également possible d'inférer des informations de haut-niveau, par rapport à l'émotion \parencite{Kobayashi2007}, l'intention \parencite{Yu2015}, et l'action \parencite{Kapsouras20141432}.

\subsection{Méthodes d'apprentissage du mouvement}
Plusieurs méthodes d'apprentissage du mouvement existent. Une de ces méthodes est l'apprentissage par la reproduction du mouvement de l'expert. Dans ce cas, l'expert fait le mouvement cible, et l'apprenant s'applique à reproduire la succession de postures composant le mouvement. La décomposition du mouvement original en postures clés permet de fragmenter l'apprentissage du mouvement en plusieurs étapes successives, facilitant ainsi la reproduction du geste par l'apprenant \parencite{Maes2012DtM}.

Il est également possible d'apprendre le geste en acquérant les propriétés biomécaniques clés, connues au préalable, du geste cible. Cela nécessite d'être capable d'identifier, de qualifier voire de quantifier les différents paramètres (rotation, déplacement, durée, vitesse, etc.) des articulations ou des membres concernées par le mouvement. Dans ce cas, il faut être en mesure de pouvoir mesurer les propriétés du mouvement de l'apprenant, afin de les comparer avec les valeurs de références, soit spécifiées par l'expert \textit{a priori}, soit directement extraites du mouvement de ce dernier.

Enfin, lorsque l'objectif du geste est la manipulation d'un objet, l'apprentissage concerne la recherche des mouvements spécifiques (et leur optimisation) permettant d'atteindre l'état ciblé de l'objet. Un exemple d'un tel apprentissage est la manipulation d'outils chirurgicaux afin d'effectuer une opération. Dans ce cas, l'objectif est d'être capable d'effectuer le geste qui permet à l'outil d'arriver à la position et l'orientation désirée, ou d'être manipulé de la façon désirée, tout en prenant en compte les contraintes de manipulation propres au corps humain.

\subsection{Apprentissage de gestes avec des d'EIAH}
Les EIAH peuvent être conçus pour apprendre des gestes à des humains. Ces systèmes proposent différentes méthodes pour faciliter l'apprentissage du mouvement : un simple affichage du geste à effectuer et à reproduire, la manipulation d'un ou de plusieurs objets, l'immersion de l'apprenant dans un environnement en réalité augmentée, virtuelle ou mixte, etc.

L'affichage du mouvement à reproduire est une méthode analogue à l'apprentissage par reproduction du mouvement. Le support étant virtuel, il est possible de fournir un affichage superposé de l'apprenant avec celui du mouvement à effectuer \parencite{Kora20151559}, de proposer une comparaison avec un ou plusieurs experts \parencite{Yoshinaga2015Doa}, ou même de permettre à l'apprenant de visualiser quelles sont les différences entre son geste et celui de l'expert \parencite{Chan2011}. Il est possible de fournir une indication de performance quant au degré de similitude du geste de l'apprenant par rapport à celui de l'expert, soit à l'aide d'un score \parencite{Maes2012DtM}, soit à l'aide d'indications de positions directement sur le geste de l'apprenant \parencite{YAMAOKA2013912}.

Lors de la manipulation d'un objet, les systèmes peuvent contrôler non seulement la position de l'objet, mais également différents paramètres propres à la situation d'apprentissage (force appliquée, position du regard lors du mouvement, durée d'une partie spécifique du mouvement, etc.) \parencite{Chellali2016Aia}. Un retour visuel peut être donné à l'apprenant, en affichant la progression de la position et l'orientation de l'objet dans une scène en temps réel \parencite{Choi2015103}, mais il est également possible de donner des retours physiques, lors de l'utilisation de dispositifs tels que les bras haptiques \parencite{Gillespie2018Hit}. La mise en situation de l'apprenant dans un environnement virtuel pour la manipulation d'un objet permet de s'affranchir des contraintes imposées par les apprentissages classiques nécessitant un environnement spécifique.

Enfin, les systèmes utilisant les environnements virtuels permettent de renforcer l'engagement des apprenants dans la tâche à accomplir. Ces environnements peuvent par exemple être utilisés dans la rééducation post-attaque cardiaque, afin de faire réaliser les mouvements répétitifs au patient tout en proposant un contexte attrayant, réaliste \parencite{Baldominos2015AAt} ou ludique \parencite{Alankus2010TCG}.

La finalité d'un EIAH dédié à l'apprentissage de gestes étant la bonne réussite de ceux-ci, il est nécessaire de pouvoir analyser les données de mouvements de l'apprenant. L'analyse des données peut se faire de manière empirique par l'expert, ou de manière automatique par le système.

Lorsqu'il n'est pas possible de formaliser les caractéristiques qui font que le geste est considéré comme étant correct, l'analyse est faite de manière empirique. Dans le sport, les différences morphologiques font qu'il n'est pas toujours possible de proposer une comparaison pertinente avec les données de l'expert \parencite{Burns2011Uvh}, bien que des travaux pallient ces problèmes en utilisant plusieurs experts \parencite{Yoshinaga2015Doa}. Dans le domaine médical, l'analyse empirique de données de mouvements reste la méthode prédominante \parencite{Chen2016TPG, Wang2013HMM, Alankus2010TCG}, car l'expertise du médecin peut être difficile à formaliser sans être contextualisée en fonction du patient et de ses antécédents, ainsi que de la progression de la rééducation.

L'analyse automatique va permettre de proposer des retours sans l'intervention systématique d'un expert. Lorsque la connaissance experte est formalisable et quantifiable, il est possible de retourner à l'apprenant non seulement le degré de réussite de son geste \parencite{Maes2012DtM}, mais également des précisions quant aux différentes parties du mouvement \parencite{YAMAOKA2013912}. L'analyse des différences entre les gestes experts et ceux de l'apprenant permet également de mettre en lumière les caractéristiques déterminantes lors de la réalisation du geste en question \parencite{Makio2007DoS}. Il est même possible de découvrir quels sont les ensembles de règles implicites utilisées par des experts pour déterminer le degré de réussite d'un geste \parencite{Pirsiavash2014AQA}.

\subsection{Analyse automatique par méthodes d'apprentissage supervisé / non-supervisé}
Il existe d'ores et déjà des EIAH qui analysent automatiquement le mouvement à l'aide de techniques de \textit{Machine Learning}. Les algorithmes sous-jacents ont pour objectif d'apprendre à séparer des données, selon des critères variables. Ce domaine comprend plusieurs familles d'algorithmes : l'apprentissage supervisé, l'apprentissage non supervisé, l'apprentissage semi-supervisé et l'apprentissage par renforcement.

L'apprentissage supervisé est un terme qui regroupe un ensemble d'algorithmes permettant d'apprendre à une machine à classer des données sous la forme d'une fonction de séparation et de prédiction construite à partir d'une base de données annotées.

L'apprentissage non-supervisé cherche à découvrir une ou des structure(s) au sein des données non-étiquetées. Les regroupements obtenus n'ont pas de sémantique associée par l'algorithme, à l'inverse de l'apprentissage supervisé. Ces regroupements se font en fonction de propriétés communes entre les différentes données fournies en entrée de l'algorithme. Les critères d'évaluations de l'efficacité des regroupements obtenus se basent sur des critères de proximité des points au sein d'un même groupe et entre les groupes (distance intra-cluster vs distance inter-cluster) \parencite{Kassab2008Fbc}.

L'apprentissage semi-supervisé fait usage d'un petit nombre de données annotées. Les avantages sont multiples : l'utilisation d'un corpus de données annotées restreint limite le coût en temps d'annotation, il est possible de comparer les groupes obtenus avec les données étiquetées, permettant de juger de la répartition des données au sein des groupes, et il est possible de donner du sens aux groupes ainsi obtenus.

L'approche supervisée pour l'analyse de mouvement requiert un corpus suffisamment grand de données étiquetées pour chaque domaine et tâche ciblés. En pratique, le temps de capture et d'annotation n'est pas adapté à une utilisation \textit{in vivo}. Dans ce contexte, l'approche semi-supervisée ou non-supervisée est la plus adaptée. En effet, avec peu de données, l'utilisation de techniques de \textit{clustering} permet de :

\begin{itemize}
	\item regrouper les gestes en fonction des propriétés communes qui peuvent être spécifiées au préalable par l'enseignant,
	\item identifier les groupes de gestes acceptables en fonction de ces propriétés,
	\item positionner le geste de l'apprenant en termes de distance par rapport à ces groupes,
	\item suivre graphiquement l'évolution de ces propriétés individuellement ou conjointement.
\end{itemize}

Ces avantages combinés font que leur utilisation est pertinente dans le cadre d'une aide à l'apprentissage \textit{in vivo}. Cependant, il n'y a que peu de travaux qui font usage de tels algorithmes pour l'apprentissage de gestes. Ainsi, l'utilisation d'algorithmes de \textit{clustering} pour faciliter l'analyse et l'apprentissage de gestes est une piste explorée dans ce travail.

\subsection{Limites des systèmes existants}
Les EIAH permettant d'assister l'apprentissage du geste et d'analyser les données de mouvement sont souvent conçus de façon \textit{ad-hoc}, c'est-à-dire créés pour une situation d'apprentissage précise, dans un contexte fixé. Il est ainsi difficile, voire impossible, de réutiliser ces EIAH dans d'autres contextes sans un travail de réingénierie conséquent. Cela mène au développement de multiples EIAH dans des domaines pourtant analogues. De plus, l'intégration de la connaissance experte dans le système est souvent réalisée de manière fortement couplée, ce qui rend la généricité difficile. L'impossibilité d'introduire de nouveaux ensembles de règles pour une autre situation d'apprentissage fait que leur portée est limitée au domaine spécifique pour lequel ils ont été développés.

L'utilisation de techniques d'apprentissage supervisé pour l'analyse des gestes pose également le problème de la constitution d'un corpus spécifique pour la tâche concernée. Il n'est pas aisé de capturer, de segmenter et d'annoter un nombre de données suffisant pour qu'un tel algorithme puisse apprendre la fonction de séparation de manière correcte. En pratique, un système réutilisable dans différents contextes devra être en mesure de proposer des algorithmes d'apprentissage automatique pouvant travailler sur un nombre restreint de données, tout en permettant d'apporter une aide suffisante pour l'expert dans sa tâche d'analyse du geste, capturables par l'expert en amont de sa tâche d'enseignement : l'utilisation des algorithmes d'apprentissage supervisé n'est donc pas adapté à un EIAH dédié à l'apprentissage de gestes dans une situation \textit{in vivo}.

Enfin, dans le cas de l'apprentissage de mouvements, la présence d'un expert permet à l'apprenant de formuler des demandes précises en terme d'observation du geste à reproduire. Dans le cas d'un apprentissage à l'aide d'un EIAH dédié, les rôles de l'apprenant et l'enseignant ne sont pas toujours les mêmes en comparaison avec une situation d'apprentissage classique. Par exemple, certains systèmes écartent totalement l'expert du processus d'apprentissage. Bien que permettant un enseignement à distance, ou diffusé de manière plus large, il n'est pas possible pour l'apprenant de questionner l'expert sur des parties spécifiques du mouvement à effectuer. Il faut également que l'EIAH soit en mesure de prendre en compte des paramètres autres que le geste (morphologie, analyse fine des mouvements de l'apprenant, observations d'autres parties du corps pouvant mener à un geste incorrect, etc.), et donc que ces paramètres soient intégrés \textit{a priori} dans le système. En pratique, les données spécifiques « annexes » (c'est-à-dire non-spécifiques au geste à réaliser) à intégrer dépendent du contexte, mais également de l'appréciation de l'expert par rapport à l'apprenant et le geste à réaliser. Ainsi, les EIAH écartant l'expert du processus peuvent mener à un apprentissage moins efficace pour l'apprenant.

\section{Problématiques de recherche}
\subsection{Verrous scientifiques}
Dans le cadre de l'utilisation d'EIAH pour l'apprentissage de gestes, notre objectif est d'améliorer à la fois la situation d'apprentissage, tant pour l'enseignant que pour l'apprenant, mais également de faciliter la réutilisation d'un tel EIAH dans des contextes variés d'apprentissage de gestes.

Ainsi, un premier verrou réside dans l'analyse « fine »  des mouvements, en vue de leur reproduction ou de la détermination de leurs propriétés acceptables. Cette analyse constitue une tâche souvent fastidieuse, contextuelle à la tâche d'apprentissage et nécessitant des connaissances scientifiques solides en géométrie, physique et biomécanique. Le plus souvent, l'apprenant et l'enseignant ne possèdent pas de telles connaissances. Les méthodes automatiques d'analyse du mouvement utilisant l'approche supervisée ne sont pas utilisables dans une situation d'apprentissage \textit{in vivo}, car ils nécessitent un nombre de données étiquetées conséquent (de l'ordre du millier de données) et spécifique à la situation d'apprentissage considérée. De plus, les différences morphologiques entre les différents acteurs de l'apprentissage doivent être prises en compte si une analyse par superposition partielle ou totale des mouvements de l'expert et de l'apprenant est envisagée notamment.

Un moyen de résoudre ce problème est d'utiliser des descripteurs cinématiques du mouvement, ces derniers étant invariants à la morphologie. Une autre méthode pour pallier ce problème réside dans l'apprentissage par observation et reproduction du mouvement, à l'aide du EIAH dédié permettant de montrer le du geste de l'expert. Bien que l'étudiant puisse s'appuyer sur une telle représentation, les propriétés acceptables du geste ne seront pas nécessairement explicitement perçues ou identifiées. Cela constitue un deuxième verrou lié à la formalisation de l'expertise de l'enseignant qui doit être intégré au système formatif sous forme de règles ou de contraintes spatiales et temporelles. Il peut être difficile, voire impossible, pour un expert d'être en mesure d'expliciter de manière quantitative les observations qu'il réalise afin de juger le geste de l'apprenant.

Un troisième verrou est lié à la représentation intelligible d'indicateurs de progression contextuels et adaptés à l'apprentissage, dépendant de l'expertise, mais aussi des capacités de perception des acteurs. En effet, bien qu'il soit possible d'extraire de nombreuses informations à partir de données de mouvements capturés brutes, un système d'aide à l'apprentissage de mouvement doit être en mesure de proposer des indicateurs pertinents en fonction de la situation d'apprentissage considérée et des objectifs pédagogiques visés. La visualisation de ces indicateurs doit être adaptée non seulement à leur type (discret ou continu), leur sémantique mais également aux besoins d'observation et d'analyse de l'enseignant.

Le quatrième verrou est lié aux retours donnés à la suite de la réalisation du geste par l'apprenant. C'est cette étape qui permet à l'apprenant d'améliorer son geste. Si ce retour est fait par un EIAH, il faut être en mesure de déterminer en amont les propriétés qui font qu'un retour est pertinent pour l'apprentissage et la progression du geste de l'apprenant. Cette connaissance est parfois tacite pour l'expert. Ces paramètres peuvent donc varier en fonction de la situation d'apprentissage considérée, mais également de l'apprenant lui-même. De plus, un système réutilisable et adaptable dans d'autres contextes faisant intervenir le mouvement ne doit pas nécessiter un processus de réingénierie lourd.

\subsection{Questions de recherche}
Le développement d'EIAH support à l'enseignement de gestes, extensibles au-delà de la tâche pour laquelle elles ont été conçues et ayant un coût minimal en termes de réingénierie, représente un défi qui soulève plusieurs questions de recherche :

\begin{itemize}
\item \textbf{Q1} : Comment développer un système permettant de caractériser le geste à l'aide de l'intégration de l'expertise d'un enseignant ?
\item \textbf{Q2} : Comment évaluer et comparer le geste (ou ses propriétés) de l'apprenant avec celui de l'enseignant afin d'évaluer la progression de l'apprentissage ?
\item \textbf{Q3} : Comment, dans une situation d'apprentissage de gestes donnée, proposer des retours pertinents et compréhensibles aux acteurs de l'apprentissage non spécialistes en analyse du mouvement ?
\end{itemize}

\subsection{Hypothèses de recherche}
À partir des éléments précédemment expliqués, ainsi que des questions de recherches formulées, six hypothèses seront vérifiées dans ces travaux :
\begin{itemize}
	\item \textbf{H1} : Il est possible de regrouper les gestes selon leurs propriétés cinématiques communes. (\textbf{Q2})
	\item \textbf{H2} : Il est possible de séparer les gestes des apprenants en deux groupes correspondant à une dichotomie geste réussi / geste raté afin de déterminer, pour une situation d'apprentissage donnée, les propriétés d'un ensemble fini de gestes réussis. (\textbf{Q1})
	\item \textbf{H3} : Il est possible de séparer les gestes en fonction de propriétés attendues et identifiées au préalable par l'expert. (\textbf{Q1})
	\item \textbf{H4} : Il est possible de corriger chaque défaut du geste de l'apprenant, en lui indiquant les défauts majeurs à corriger en premier. Un défaut majeur est identifié par la plus grande distance séparant le mouvement courant de l'apprenant, du groupe de gestes acceptables ayant éliminé ce défaut. (\textbf{Q2})
	\item \textbf{H5} : L'utilisation du système MLA basée sur l'hypothèse 4 permet d'améliorer l'apprentissage du geste par rapport à une situation sans le système MLA. (\textbf{Q3})
	\item \textbf{H6} : L'utilisation du système MLA en tant qu'assistant à l'enseignant permet d'améliorer l'apprentissage du geste par rapport à une situation sans MLA, et à une situation avec MLA et sans enseignant. (\textbf{Q3})
\end{itemize}

La validation ou non de ces hypothèses s'effectuera à travers le système d'aide à l'apprentissage de mouvements développé au cours de cette thèse.

\section{Aperçu des contributions}
\subsection{Motion learning Analytics (MLA)}
Afin de répondre à ces questions, notre approche s'est axée autour du système \textit{Motion Learning Analytics} (MLA), développé pour les besoins de cette thèse. Il s'agit d'un système d'analyse de mouvement, combinant la captation, le filtrage, le traitement et l'analyse (à l'aide de méthodes d'apprentissage non-supervisé ou semi-supervisé) du mouvement. L'approche consiste à traiter des données de mouvement brutes, afin de les rendre : (i) exploitables, en éliminant les défauts inhérents au système de capture utilisé et (ii), compréhensibles, en extrayant des descripteurs de mouvement à partir des données filtrées. Une fois ces étapes réalisées, l'objectif du système est alors double : (i) proposer automatiquement une évaluation et des conseils pour l'amélioration du geste de l'apprenant à partir de la comparaison de propriétés du mouvement de l'apprenant avec celles attendues et calculées à partir des mouvements de l'expert, et (ii) proposer un affichage graphique de la progression de l'apprenant, en fonction des défauts considérés, sur lequel peut s'appuyer l'expert.

Cette plateforme a permis de valider notre approche, et d'apporter ainsi une preuve de concept de notre chaîne allant de l'acquisition du mouvement jusqu'à l'analyse de ces derniers :

\begin{itemize}
    \item L'acquisition du mouvement se fait à l'aide du Perception Neuron, une combinaison constituée de multiples capteurs inertiels, permettant de capturer le mouvement en temps réel.
    \item Les données exportées au format BVH sont passées dans une chaîne de traitements, afin d'améliorer le signal. Les données sont ensuite utilisées afin de calculer des descripteurs, correspondant à des aspects spécifiques du mouvement (vitesse, accélération, etc.).
    \item Les descripteurs sont utilisés dans un processus de \textit{clustering}, afin d'étudier la séparation possible des gestes, selon leur degré de réussite, leurs défauts ou les propriétés d'acceptabilité identifiées en amont par un expert.
    \item Les données de l'apprenant sont comparées à celles de l'expert, afin de déterminer quels sont les conseils à donner à l'apprenant afin qu'il améliore son geste.
\end{itemize}

\subsection{Traitement et segmentation des mouvements}
La chaine de traitement des données permet non seulement de filtrer les artéfacts propres aux données obtenues à partir du Perception Neuron, mais également de n'en garder que la partie jugée utile à l'aide d'un algorithme de segmentation et d'en extraire des descripteurs jugés pertinents dans la situation d'apprentissage concernée.

\subsection{Séparation des données de mouvement et sémantique associée}
L'utilisation d'algorithmes de \textit{clustering} permet non seulement de travailler avec un nombre restreint de données, mais également de proposer un regroupement des données en fonction de propriétés définies au préalable par l'expert. L'ajout d'un petit nombre de données étiquetées permet de donner un sens aux groupements obtenus, permettant une comparaison des données de l'apprenant par rapport aux gestes jugés bons ou mauvais.

\subsection{Système de retour visuel}
Enfin, un système de retour visuel, permettant à l'utilisateur de juger de la distance des données de l'apprenant par rapport à celles de l'expert, permet de consolider l'avis de l'expert ou de donner des pistes quant aux défauts les plus significatifs du geste de l'apprenant, dans le cas où le jugement est difficile.

\section{Expérimentations}
Le système a été testé au travers de trois expérimentations, permettant chacune de tester les hypothèses présentées dans cette introduction. Ces trois expérimentations consistent en la réalisation de gestes dont deux exigent une certaine technicité, puis à l'utilisation de ces données pour tester les hypothèses. La première expérimentation consistait à lancer une balle dans deux corbeilles éloignées l'une de l'autre. L'objectif ici était de voir s'il était possible d'obtenir une séparation acceptable du geste en fonction d'une ou plusieurs propriétés définies au préalable, correspondant à différentes stratégies de lancers. Une séparation correspondant au type de lancer a été obtenue. Le geste de la deuxième expérimentation était celui du \textit{Bottle Flip Challenge}, où il faut lancer une bouteille tout en lui imprimant une rotation sur son axe horizontal, puis la faire retomber debout sur une surface plane. Les données de cette expérimentation ont été utilisées afin de vérifier s'il était possible de séparer les gestes en deux groupes correspondant aux gestes réussis et ceux ratés. Il a été possible d'obtenir une bonne séparation mais pas une séparation en fonction des groupes considérés. Enfin, la troisième expérimentation s'est appuyée sur le lancer de fléchettes comme domaine applicatif, afin de tester la pertinence du système MLA en tant qu'outil de regroupements des gestes suivant des propriétés identifiées en amont par un expert, et en tant qu'outil de retours à l'apprenant et d'assistance à l'enseignant.

\section{Plan du manuscrit}
Le chapitre suivant porte sur un état de l'art des EIAH dédiés à l'apprentissage du geste. Y sont introduit les différentes modalités utilisées au sein de ces EIAH : objectif du système, données utilisées et leur représentation informatique, méthodes de captation, analyses effectuées, retours fournis à l'apprenant et place de l'expert dans le processus.\\

Le chapitre trois propose une étude bibliographique des informations extractibles du mouvement, appelées descripteurs du mouvement. Ces descripteurs permettent de caractériser le mouvement afin de pouvoir l'analyser en fonction des besoins d'observations ou des propriétés cinématique du mouvement. Une analyse bibliographique des différents descripteurs, ainsi que de leur utilisation dans des contextes de recherche sur le mouvement est réalisée. Un tableau présentant des descripteurs couvrant un ensemble de propriétés observables est proposé à la fin de ce chapitre.\\

Le chapitre quatre se concentre sur les méthodes d'analyses appliquées à des mouvements capturés. À partir des descripteurs calculés, il est possible de réaliser des analyses précises sur le mouvement, soit par un expert, soit par un processus automatique. Ce chapitre présente ces analyses sous deux angles majeurs : l'analyse empirique qualitative et l'analyse automatique quantitative.\\

Le chapitre cinq présente la contribution apportée au travers du système MLA, développé pour l'aide à l'évaluation de gestes humains. Y sont présentées les différentes étapes, du traitement des données des mouvements brutes jusqu'à l'aide proposée à l'expert et aux apprenants.\\

Le chapitre six est dédié aux trois expérimentations qui ont permis de tester les hypothèses de recherche.\\

Pour terminer, le dernier chapitre fournit un résumé des apports de cette thèse, et ouvre plusieurs pistes et perspectives de recherche.

