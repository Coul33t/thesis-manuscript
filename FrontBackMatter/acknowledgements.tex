%*******************************************************
% Acknowledgments
%*******************************************************
\pdfbookmark[1]{Remerciements}{remerciements}
\chapter*{Remerciements}


Mes remerciements à \textbf{Sébastien George} et \textbf{Ludovic Hamon}, pour m'avoir accepté en thèse, pour l'encadrement de cette thèse, la bonne humeur, les conseils toujours pertinents, la patience, le soutien tout au long des étapes de cette thèse, ainsi que pour m'avoir apporté la rigueur qui me faisait tant défaut avant d'arriver en thèse.\\


Je souhaite remercier également l'équipe du LITIS, qui m'a initié au monde de la recherche, et plus particulièrement messieurs \textbf{Sébastien Adam}, \textbf{Pierre Héroux} et \textbf{Laurent Heutte}, pour m'avoir chacun à leur manière fait découvrir la recherche et donné une forte envie et motivation pour continuer dans ce domaine.\\

Merci à mes collègues doctorants, ex-doctorants, docteurs ou post-doctorants : \textbf{Guillaume}, \textbf{Inès}, \textbf{Damien}, \textbf{Vincent}, \textbf{Aïcha}, \textbf{Aous}, \textbf{Esteban}, \textbf{Zeyneb}, \textbf{Jean}, \textbf{Oussema}, \textbf{Claire}, \textbf{Dalal}, \textbf{Ibtissem}, \textbf{Pierre-Yves} et tous les autres que j'oublie peut-être (pardon), pour avoir fait du temps passé au $\text{CERIUM}^2$ un agréable moment. Je souhaite remercier particulièrement \textbf{Inès Dabbebi} pour m'avoir supporté pendant 2 longues années, moi, ma musique, mes thés, ma nourriture dans le bureau, mes gros mots, ainsi que \textbf{Guillaume Loup}, pour sa gentillesse, son aide et son soutien autant technique que moral, et de manière générale pour avoir été mon parrain de thèse pendant ces quelques années passées ici. On notera également la patience légendaire dont il a dû faire preuve pour me supporter pendant son temps ici, car il n'a jamais pu se débarrasser de moi, même quand il a déménagé en info, où je l'ai suivi (pour le meilleur, et pas vraiment de pire au final). Les loooooo(...)ooongues conversations que nous avons (souvent) eu tous les trois resterons un souvenir très agréable pour moi.\\

Merci à tous les membres de \textbf{l'équipe IEIAH} et plus largement du \textbf{LIUM} pour leur accueil. Je remercie également tous mes collègues de \textbf{l'IUT de Laval}, en particulier ceux du \textbf{département Informatique}. Un grand merci au \textbf{service administratif de l'IUT de Laval} également, grâce à qui toutes mes démarches administratives se sont toujours très bien passées, ainsi qu'à \textbf{Elisabeth} et \textbf{Marie}, nos deux secrétaires du département informatique.\\


Merci à \textbf{Yann Walkowiak} et \textbf{Pierre Laforcade} pour les conversations intéressantes et les blagues d'un genre douteux comme je les aime, \textbf{Clément Laborie} pour avoir été mon bae ici (et je pense qu'il peut aussi me remercier pour avoir été son exutoire contre la SNCF), et enfin à \textbf{Nathalie Vieillard} pour avoir été ma mère adoptive de l'IUT de Laval.\\

Depuis petit, j'ai toujours voulu enseigner. Lors de ma thèse, j'ai eu la chance de pouvoir donner des cours, et je tenais à remercier l'ensemble des promotions auxquelles j'ai enseigné, car ça n'a été que du plaisir (même si j'ai parfois \textit{légèrement} haussé la voix). Plus particulièrement, j'aimerai remercier \textbf{Hugo Morali}, \textbf{Florian Pellegrin}, \textbf{Antoine Lambert}, \textbf{Clément Fievez}, \textbf{Alexis Nolat}, \textbf{Maëliss Coué}, \textbf{Justin Martin}, \textbf{Bibite}, \textbf{Quentin Pineau}, \textbf{Evan Delaunay}, \textbf{Euntoine}, \textbf{Florian Plaut} et une bonne partie des sous-doués de la promo 2018-2020 (ça fait un peu trop de monde, pas envie de tout écrire).\\

Sans \textbf{Valval}, \textbf{Princess Mayva}, \textbf{Léo the tasty virgin}, \textbf{Queen Juju} et bien évidemment \textbf{Marie Sproutch Bouin}, la 3ème année de thèse n'aurait pas été aussi bien que ce qu'elle a été. Un grand merci à eux pour les nombreuses visites et excellents moments passés (ainsi que les innombrables thés), autant dans le bureau qu'en dehors de l'IUT.\\

Merci également à tous mes amis restés sur Rouen, pour m'avoir permis de passer des moments agréables quand je rentrais : \textbf{PH}, \textbf{Fourré}, \textbf{Dodu}, \textbf{Dandelinos}, \textbf{Kiki}, \textbf{Guigui}, et tous les autres. Même s'ils ne sont pas à Rouen, je tiens à remercier plusieurs autres personnes, à commencer par \textbf{Bastien Resse}, pour être l'homme de classe et de culture qu'il est, ayant même réussi à m'intéresser à la politique sous un angle neutre. Également le tenant de ma maison sur Paris, \textbf{Freddy " DiFFtY "}, qui me supporte depuis beaucoup trop longtemps pour quelqu'un qui serait sain d'esprit, ainsi que \textbf{Chacha la dame du CDI}, deux personnes avec qui j'ai toujours passé d'excellents moments (et dieu sait que je n'aime pourtant pas Paris). Je remercie également (sous la pression) \textbf{Lussie Leroy}, bien que ce soit plutôt à elle de me remercier au final, vu les efforts que je doit déployer pour la supporter au quotidien, elle et ses excentricités.\\

Je pense que sans \textbf{PH} et \textbf{Dandelinos} en vocal presque tous les soirs depuis que je suis arrivé à Laval, je n'aurai pas tenu aussi longtemps sans devenir fou dans cette ville. Aussi, il me semble naturel de les remercier tout particulièrement pour les discussions distinguées, intelligentes et matures (150 de QI à nous trois), les games de LoL, le coaching de qualité sur le ladder SC2 et enfin pour les games de CS{\string:}GO où je carry toute l'équipe.\\


Je remercie ma \textbf{famille} pour ses encouragements, et en particulier mes \textbf{parents} pour m'avoir toujours soutenu lors de mes études et avoir été fier de moi, même quand j'ai parfois pu faire des choix à première vue un peu douteux. Sans leur soutien moral ainsi que financier (car oui, ça coûte des sousous de louer un appartement, s'inscrire à la fac, \textit{etc.}), rien de tout ça n'aurait été possible.\\

\vspace{0.5cm}
Enfin, je voudrais terminer par une citation qui m'a inspiré, et qui devrait tous nous inspirer, à toujours aller de l'avant quelles que soient les difficultés rencontrées :
\vspace{0.25cm}\\
« \textit{Qu'est-ce que tu crois qu'il aurait fait Toto à ta place, tu crois qu'il aurait lâché l'affaire, qu'il aurait abandonné comme ça Toto, hein ? Non Toto il aurait continué, ouais Toto il aurait continué ouais.} »